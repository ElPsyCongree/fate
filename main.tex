\documentclass{article}        %导入xeCJK包
\usepackage{CJKutf8}
\usepackage{ruby}
\usepackage{graphicx}

\title{Fate Stay Night} % Article title
\author{}
\date{}

\begin{document}

\maketitle

\fontsize{15}{23}
\selectfont

\begin{CJK*}{UTF8}{gbsn}
 \begin{center}序章\end{center}

\end{CJK*}

\begin{CJK}{UTF8}{min}
それは、\ruby{稲}{いな}\ruby{妻}{づま}のような\ruby{切}{き}っ\ruby{先}{さき}だった。

\ruby{心}{しん}\ruby{臓}{ぞう}を\ruby{串}{くし}\ruby{刺}{ざ}しにせんと\ruby{繰}{く}り\ruby{出}{だ}される\ruby{槍}{やり}の\ruby{穂}{ほ}\ruby{先}{さき}。

\ruby{躱}{かわ}そうとする\ruby{試}{こころ}みは\ruby{無}{む}\ruby{意}{い}\ruby{味}{み}だろう。

それが\ruby{稲}{いな}\ruby{妻}{づま}である\ruby{以}{い}\ruby{上}{じょう}、\ruby{人}{ひと}の\ruby{目}{め}では\ruby{捉}{と}えられない。

だが。

この\ruby{身}{み}を\ruby{貫}{つらぬ}こうとする\ruby{稲}{いな}\ruby{妻}{づま}は、

この\ruby{身}{み}を\ruby{救}{すく}おうとする\ruby{月光}{げっこう}に\ruby{弾}{はじ}かれた。

しゃらん、という\ruby{華}{か}\ruby{麗}{れい}な\ruby{音}{おと}。

\ruby{否}{いや}。\ruby{目}{もく}\ruby{前}{ぜん}に\ruby{降}{お}り\ruby{立}{た}った\ruby{音}{おと}は、\ruby{真}{しん}\ruby{実}{じつ}\ruby{鉄}{てつ}よりも\ruby{重}{おも}い。

およそ\ruby{華}{はな}やかさとは\ruby{無}{む}\ruby{縁}{えん}であり、\ruby{纏}{ま}った\ruby{鎧}{よろい}の\ruby{無}{む}\ruby{骨}{こつ}さは\ruby{凍}{い}てついた\ruby{夜}{や}\ruby{気}{き}そのものだ。

\ruby{華}{か}\ruby{美}{び}な\ruby{響}{ひび}きなど\ruby{有}{あ}る\ruby{筈}{はず}がない。

\ruby{本}{ほん}\ruby{来}{らい}\ruby{響}{ひび}いた\ruby{音}{おと}は\ruby{鋼}{はがね}。

ただ、それを\ruby{鈴}{すず}の\ruby{音}{おと}と\ruby{変}{か}えるだけの\ruby{美}{うつく}しさを、その\ruby{騎}{き}\ruby{士}{し}が\ruby{持}{も}っていただけ。

「\ruby{問}{と}おう。\ruby{貴}{あ}\ruby{方}{なた}が、\ruby{私}{わたし}のマスターか」

\ruby{闇}{やみ}を\ruby{弾}{はじ}く\ruby{声}{こえ}で、\ruby{彼}{か}\ruby{女}{のじょ}は\ruby{言}{い}った。

「\ruby{召}{しょう}\ruby{喚}{かん}に\ruby{従}{したが}い\ruby{参}{さん}\ruby{上}{じょう}した。

これより\ruby{我}{わ}が\ruby{剣}{つるぎ}は\ruby{貴}{あ}\ruby{方}{なた}と\ruby{共}{とも}にあり、\ruby{貴}{あ}\ruby{方}{なた}の\ruby{運}{うん}\ruby{命}{めい}は\ruby{私}{わたし}と\ruby{共}{とも}にある。こに、\ruby{契}{けい}\ruby{約}{やく}は\ruby{完}{かん}\ruby{了}{りょう}した」

そう、\ruby{契}{けい}\ruby{約}{やく}は\ruby{完}{かん}\ruby{了}{りょう}した。

\ruby{彼}{か}\ruby{女}{のじょ}がこの\ruby{身}{み}を\ruby{主}{あるじ}と\ruby{選}{えら}んだように。

きっと\ruby{自}{じ}\ruby{分}{ぶん}も、\ruby{彼}{か}\ruby{女}{のじょ}の\ruby{助}{たす}けになると\ruby{誓}{ち}ったのだ。

\ruby{月光}{げっこう}はなお\ruby{冴}{さ}え\ruby{冴}{ざ}えと\ruby{闇}{やみ}を\ruby{照}{て}らし。

\ruby{土}{ど}\ruby{蔵}{ぞう}は\ruby{騎}{き}\ruby{士}{し}の\ruby{姿}{すがた}に\ruby{倣}{なら}うよう、かつての\ruby{静}{しず}けさを\ruby{取}{と}り\ruby{戻}{もど}す。

\ruby{時}{じ}\ruby{間}{かん}は\ruby{止}{と}まっていた。

おそらくは\ruby{一}{いち}\ruby{秒}{びょう}すらなかった\ruby{光}{こう}\ruby{景}{けい}。

されど。

その\ruby{姿}{すがた}ならば、たとえ\ruby{地}{じ}\ruby{獄}{ごく}に\ruby{落}{お}ちようと、\ruby{鮮}{せん}\ruby{明}{めい}に\ruby{思}{お}い\ruby{返}{かえ}す\ruby{事}{こと}ができるだろう。

\ruby{僅}{わず}かに\ruby{振}{ふ}り\ruby{向}{む}く\ruby{横}{よこ}\ruby{顔}{がお}。

どこまでも\ruby{穏}{おだ}やかな\ruby{聖}{せい}\ruby{緑}{みどり}の\ruby{瞳}{ひとみ}。

\ruby{時}{じ}\ruby{間}{かん}はこの\ruby{瞬}{しゅん}\ruby{間}{かん}のみ\ruby{永}{えい}\ruby{遠}{えん}となり、\ruby{彼}{か}\ruby{女}{のじょ}を\ruby{象}{しょう}\ruby{徴}{ちょう}する、\ruby{青}{あお}い\ruby{衣}{ころも}が\ruby{風}{かぜ}に\ruby{揺}{ゆ}れる。

\ruby{差}{さ}し\ruby{込}{こ}むのは\ruby{僅}{わず}かな\ruby{蒼}{そう}\ruby{光}{こう}。

\end{CJK}

\end{document}